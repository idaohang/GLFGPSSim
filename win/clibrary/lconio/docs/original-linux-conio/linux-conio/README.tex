% This LaTeX-file was created by <fractor> Sun Mar  9 00:01:02 1997
% LyX 0.8 (C) 1995 by Matthias Ettrich
\nonstopmode
\documentclass[10pt,a4paper,oneside,onecolumn]{article}
\usepackage{lyx}
\usepackage[american]{babel}
\usepackage{helvet}
\pagestyle{plain}
\setcounter{secnumdepth}{3}
\setcounter{tocdepth}{3}
\begin{document}


\subsubsection*{Mental EXPlosion proudly presents:}


\vspace*{\fill}
\lyxtitle{Linux-conio a conio.h for Linux.}
\vspace*{\fill}


\subsection*{\centering Version 1.02}

\tableofcontents


\section{What it is}

The conio.h is a header file for a library which is used for a lots
of keyboard input and screen output functions under DOS C-compilers.
Linux has got the curses library to do this job. But the curses library
is somewhat incompatible to the conio functions. If you want to port
a text-based DOS program which was written in C or C++ to Linux you'll
be surprised of the lots of work it costs!

Well, we did the work for you - here you have a nearly complete implementation
of the DOS conio.h for Linux!

If you don't know by now what that means, you can forget this an de-install
this thing - this is just for people needing it.

If you want to start C or C++ programming only for Linux - then start
to learn how to use the curses library!


\section{Installation and how to use it.}


\subsection{Installation}

First of all: Install ncurses! 

(You can get it from: ftp.netcom.com /pub/zm/zmbenhal/ncurses) This
version of the conio.h was tested with ncurses 1.9.9e - there are
known problems with earlier version of ncurses - use Linux-conio 1.00
if you have troubles compiling conio.h with an earlier version of
ncurses !

Do a \lyxquote{}make\lyxquote{} followed by a \lyxquote{}make install\lyxquote{}
and do what it says. If the \lyxquote{}make\lyxquote{} fails edit
the Makefile and set the correct path to the variable CURSESDIR -
the default is: CURSESDIR = /usr/local


\subsection{How to use it}

When the files libconio.a etc... have been created and installed you
can now try to port a DOS text-based program to Linux. Please make
sure that no other operating system specific functions are used in
there such as functions of the libraries \lyxquote{}dos.h\lyxquote{}
or \lyxquote{}i86.h\lyxquote{}. To compile the program you must
type something like:

\begin{quote}

gxx -LCURSESDIR -LCONIODIR -ICONIODIR -ICURSESDIR FILE -lconio -lncurses


\end{quote}

For example:

\begin{quote}

gcc -L/usr/local/lib -I/usr/local/include/ncurses -I/usr/local/include
myprog.c -lconio -lncurses

\end{quote}


\subsection{Some technical information:}


\subsubsection{Implemented functions}

All functions defined in the conio.h of Watcom C++ 10.x are implemented.

All functions defined in the conio.h of Borland C++ 3.10 are implemented
except:

\begin{itemize}

\item puttext() 

\item gettext() 

\item movetext()

\end{itemize}


\subsubsection{Differences between the Linux-conio and DOS based conio.h's}


\paragraph{A) }

Before using any function of the Linux-conio please call initconio()
!

This is done automatically when calling a conio-function uninitialized,
but it's better if you do this at the start of the program ! Please
also call doneconio() before the end of your program - this is not
done automatically, but makes sure that the terminal settings are
reset to normal state.

If you only use inp(), inpw(), inpd(), outp(), outpw() and / or outpd()
you won't have to call initconio() or doneconio() !

\begin{description}

\item [ATTENTION:]In order to use the inp etc... and outp etc... functions
you must be superuser when calling the program ! (Or the superuser
must \lyxquote{}chmod +s\lyxquote{} the executable !)

\end{description}


\paragraph{B)}

Please be careful when using textattributes !

They may differ from DOS-standard ! I've tried a lot to make them
equal, but I am not sure - please try them before using them. 

That is especially meant for the function textattr() and for the return
value inforec-\(>\)attribute of gettextinfo(). The return value
inforec-\(>\)normattr of gettextinfo will be garbage ! Please try
not to use it - this is not a bug, but there just no technical possibility
to implement it, because we can't know the values of the colorpairs
before starting conio.h !


\paragraph{C)}

The user of your application will get a warning message if his terminal
does not support colors.

If you don't want this, set the variable color\_warning to 0 before
calling initconio().


\paragraph{D) }

The window() function should only be used for ONE window, since ncurses'
window conception is totally different from DOS. A better idea is
to create subwindows to conio\_scr. You must use the ncurses functions
directly for this.


\section{Copyright, author, bug reports etc...}

The Linux-conio.h is copyright (c) 1996,97 by Fractor / Mental EXPlosion.
Look at our home page to get more information on Mental EXPlosion


http://www.obh.snafu.de/\~{}marte/friends/mxp.html

The Linux-conio can be copied and used freely as described in the
GNU LIBRARY PUBLIC LICENSE Version 2.00 (GLPL).

\begin{description}

\item [ATTENTION:]As the Linux-conio bases on ncurses you must also follow
the copyright terms of ncurses when using the conio.h! 

\end{description}

If you find a bug, report it to: fractor@germanymail.com If you have
no Internet please report it to: 

Gerald Friedland @ 2:2410/252 in Fido Net (But response may take long...)

Questions, enhancements, etc... also go to: fractor@germanymail.com

Special thanx go to: Rich Cochran and Mark Hahn for finding a solution
fot the inpw/inpd bug.

Trademarks etc... are trademarks of their owners and are not specially
marked as such.

Linux is free - please fell free to use and support it !

Remember: Any ported program to Linux is better than its DOS equivalent!

\end{document}
